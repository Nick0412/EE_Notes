\subsection[Introduction to Circuits]{Introduction to \\ Circuits}

\subsubsection{Basic Components}
\stitle{Resistors}
The resistor is one of the most fundamental building blocks of analog circuits.
We denote a resistor with resistance $R$ as follows.
\begin{figure}[H]\centering
\begin{circuitikz}
	\draw (0,0) to[R=$R$] (2,0);
\end{circuitikz}
\end{figure}
Resistance has units of Ohms and the SI symbol for its units is $\Omega$.
When studying resistors, we usually want to relate the voltage across the resistor's terminals to the current passing through it.
The relationship is described by Ohm's Law.
\begin{mdframed}[backgroundcolor=frameColor,linecolor=borderColor,linewidth=2pt,roundcorner=8pt,align=center]
\textbf{Ohm's Law} relates the voltage $V$ and the current $I$ through a resistor with resistance $R$ by the equation.
\[
	V = IR
\]
Represented by the following schematic.
\begin{figure}[H]\centering
\begin{circuitikz}[american voltages]
	\draw (0,0) to[R=$R$,f=$I$,v_>=$V$, o-o] (2,0);
\end{circuitikz}
\end{figure}
\end{mdframed}
Voltages has units of volts and its units are represented by $V$. 
Current has units of Amperes and its units are represented by $A$.

\stitle{Capacitors}
A capacitor is typically a set of parallel plates that are separated by some insulating material such as air.
We denote a capacitor with capacitance $C$ as follows.
\begin{figure}[H]\centering
\begin{circuitikz}
	\draw (0,0) to[C=$C$] (2,0);
\end{circuitikz}
\end{figure}
Capacitors work by storing some positive charge on one of the plates and negative charge on the opposite plate.
This causes an electric field between the plates and this is where the capacitor stores its energy.
We relate the capacitor's capacitance with its voltage as follows.
\begin{mdframed}[backgroundcolor=frameColor,linecolor=borderColor,linewidth=2pt,roundcorner=8pt,align=center]
Given the following capacitor with capacitance $C$ and charge $Q$ on one of the plates
\begin{figure}[H]\centering
\begin{circuitikz}[american voltages]
	\draw 
		(0,0) to node[above,pos=1,o-]{$+Q$} (.5,0)
		to [C=$C$] (1.5,0) to node[above,pos=0]{$-Q$} (2,0)
		(0,0) to [open,v_>=\raisebox{-.2in}{$V$},o-o] (2,0);
\end{circuitikz}
\end{figure}
we relate the voltage $V$ across the capacitor with the equation.
\[
	V = \frac{Q}{C}
\]
\end{mdframed}


